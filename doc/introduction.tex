\newpage
\section{Introduction}
\label{sec:introduction}

\subsection{Contexte du projet}

% Un paragraphe naturel avec un saut de ligne
\paragraph{} La bourse est un marché organisé où s’échangent des actions, des obligations et d’autres instruments financiers. C’est un espace où entreprises, investisseurs et institutions se croisent, cherchant soit à lever des fonds, soit à faire fructifier leur capital. Les prix y sont déterminés par l'offre, la demande, mais aussi par l’anticipation des événements futurs.

\paragraph{} Aujourd’hui, cet équilibre est plus instable que jamais. Les tensions commerciales entre les États-Unis et la Chine, sous l’impulsion de décisions politiques imprévisibles comme celles de Donald Trump par le passé, continuent de peser sur les marchés mondiaux. 


\subsection{Objectif du projet}

\paragraph{} Aujourd'hui, comprendre les mécanismes de la bourse devient essentiel.
Notre projet propose un simulateur de bourse permettant à l’utilisateur de s'initier aux dynamiques financières, sans risque financier réel, en prenant en compte non seulement l'évolution naturelle des marchés mais aussi l'impact d’événements économiques majeurs.
\paragraph{} Grâce à ce simulateur, l’utilisateur peut expérimenter différentes stratégies d’investissement, voir les conséquences de ses choix, et apprendre à composer avec les aléas du marché.
\paragraph{}Nous cherchons à former une intuition boursière, sans passer par la case “perte réelle”.

\subsection{Organisation du rapport}

\paragraph{} Ce rapport s’ouvre sur une présentation des notions essentielles liées à la bourse et des contraintes que nous avons retenues pour encadrer le projet.
\paragraph{}Nous détaillons ensuite la conception du simulateur, son architecture logicielle et l’interface utilisateur envisagée.
Un manuel d’utilisation est proposé pour guider la prise en main de l’application.
\paragraph{}Enfin, nous revenons sur le déroulement du projet avant de conclure par un bilan du travail accompli et des améliorations possibles.