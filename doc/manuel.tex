\newpage
\section{Manuel utilisateur}
\label{sec:manuel}



Cette section présente le manuel d’utilisation du simulateur de bourse.  
Elle décrit les principales fonctionnalités disponibles à travers l’interface graphique, et explique comment naviguer et interagir avec l’application.  
Des captures d'écran du logiciel y sont intégrées afin d'illustrer concrètement les différentes actions possibles.

\subsection{Lancement de l'application}

Pour lancer le simulateur, il suffit d’exécuter la classe \texttt{Test} (contenue dans le package \texttt{test}), qui instancie la fenêtre principale (\texttt{MainWindow}).  
Au démarrage, l’interface affiche la liste des entreprises disponibles et permet à l'utilisateur de commencer immédiatement la simulation.

\fig{images/lancement.png}{12cm}{8.25cm}{Fenêtre principale au lancement du simulateur}{lancement}

\vspace{8.5cm}
\subsection{Création d'entreprise depuis la fenêtre principale}

Depuis la fenêtre principale, l’utilisateur a la possibilité de créer de nouvelles entreprises afin d’enrichir le marché boursier simulé.  
Un bouton dédié ouvre une fenêtre de saisie où l'utilisateur peut définir les caractéristiques de l'entreprise : nom, secteur d'activité, capital initial, volatilité et nombre d'actions disponibles.

Cette fonctionnalité permet de personnaliser l’environnement de simulation et d'observer l'impact de nouvelles entreprises sur l’évolution du marché.

\figk{images/creation.png}{12cm}{8.25cm}{Création d'une entreprise depuis la fenêtre principale}{creation_entreprise}




\vspace{0.5cm}
\subsection{Consultation du marché}

L’utilisateur peut consulter la liste des actions et obligations disponibles via l’onglet "Liste des Actions".  
Chaque actif est présenté avec ses principales caractéristiques (nom, prix actuel, variation, secteur) comme vu sur la figure \ref{fig:lancement}.


\vspace{0.5cm}
\subsection{Gestion du portefeuille}

Dans l’onglet "Mon Portefeuille", l’utilisateur peut suivre les actifs détenus, consulter les statistiques d’investissement (répartition par secteur) et visualiser l’historique des transactions.

\figk{images/portefeuille.png}{12cm}{8.25cm}{Portefeuille utilisateur : actifs détenus et performances}{portefeuille}

On peut également consulter les statistiques de notre portefeuille en cliquant sur la rubrique "Statistiques" 

\figk{images/stats.png}{12cm}{8.25cm}{Portefeuille utilisateur : affichage des statistiques}{stats}



\vspace{3.5cm}
\subsection{Réaliser des transactions}

Depuis la liste des actions ou directement dans le portefeuille, l'utilisateur peut acheter ou vendre des actifs.  
Des menus contextuels simplifient les opérations (clic droit → acheter/vendre).

\figk{images/transac.png}{12cm}{8.25cm}{Exemple d'achat d'une action via le menu contextuel}{transaction}

\figk{images/vente_actif.png}{12cm}{8.25cm}{Vente d'un actif depuis le portefeuille}{vente_actif}


\vspace{0.5cm}
\subsection{Visualisation graphique}

L’onglet "Graphiques" permet de suivre l'évolution des prix par entreprise et la performance par secteur sous forme de courbes et d’histogrammes dynamiques.

Il est possible de filtrer l'affichage des graphiques selon un secteur spécifique en utilisant le menu déroulant prévu à cet effet.  
Ainsi, l’utilisateur peut concentrer son analyse sur un domaine d'activité particulier, par exemple uniquement les entreprises du secteur technologique ou énergétique.

\figk{images/graph.png}{12cm}{8.25cm}{Graphique de suivi de l'évolution des prix}{graphique}
\vspace{0.5cm}
\subsection{Simulation automatique}

L’utilisateur peut activer la simulation automatique pour faire évoluer le marché sans intervention manuelle, en utilisant les boutons prévus dans l'interface de contrôle.

\figk{images/portefeuille.png}{12cm}{8.25cm}{Portefeuille utilisateur : actifs détenus et performances}{portefeuille}
\vspace{0.5cm}

\paragraph{} L'objectif est de réussir a investir pour faire fructifier votre votre solde de départ. L'utilisateur est libre d'adopter différentes stratégies : investissement massif sur un seul secteur, diversification du portefeuille, ou spéculation sur les évolutions rapides des cours.  
L'outil de simulation permet de tester et d'affiner ses décisions en temps réel, dans un environnement contrôlé.