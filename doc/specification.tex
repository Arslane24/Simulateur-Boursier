\newpage
\section{Spécification du projet}
\label{sec:specification}

\paragraph{}Nous avons exposé l’objectif général du projet dans la  section \ref{sec:introduction}.
Dans cette partie, nous détaillons la spécification fonctionnelle de notre application.
Elle s’appuie principalement sur le cahier des charges et définit les concepts de base, les contraintes à respecter, ainsi que les principales fonctionnalités attendues.



\subsection{Notions de base et contraintes du projet}
\label{sec:spec1}

\subsubsection{Fonctionnement général du logiciel}

\paragraph{}Le logiciel simule un environnement boursier interactif où l’utilisateur peut gérer un
portefeuille financier en temps réel, avec les mécanismes suivants.
\paragraph{Initialisation de la simulation :}



Le système crée un marché boursier virtuel composé de N entreprises
configurées avec leurs caractéristiques financières : prix d'action initial, nombre
d'actions disponibles et obligations associées (taux fixes ou variables). Les données peuvent être générées aléatoirement ou chargées depuis un fichier CSV. 

L'utilisateur a la possibilité d'ajouter manuellement des entreprises en précisant
leur secteur, taille et niveau de risque. Le module calcule automatiquement les
indicateurs clés (volatilité, volume d'échanges) pour établir un environnement
initial de marché réaliste et personnalisable.

\paragraph{Boucle principale de simulation :}
Notre système repose sur une boucle de traitement itérative qui pilote la
dynamique de marché.

À chaque cycle, le moteur de simulation :
\begin{enumerate}
    \item Génère des événements aléatoires.
\item Recalcule l'ensemble des prix d’actions.
\item Met à jour les portefeuilles utilisateur.
\end{enumerate}

\paragraph{Mise à jour de bourse :} Selon les évènements générés, il y aura une mise à jour de la bourse en termes de
prix d’actions ce qui permet une évolution progressive et réaliste du marché, tout
en gardant une trace complète des fluctuations.

\paragraph{Interactions utilisateur :}L'utilisateur peut acheter ou vendre des actions et obligations à tout moment.
Toutes les transactions sont enregistrées dans son portefeuille, avec un suivi en
temps réel des performances et historique des opérations. Le système affiche
l'impact des décisions sur la valeur du portefeuille et les conditions du marché.

\paragraph{Visualisation des fluctuations et les indicateurs
économiques :} Le système propose une interface interactive affichant en temps réel les
fluctuations boursières sous forme de graphiques dynamiques (cours
historiques, tendances sectorielles). Les principaux indicateurs financiers (P/E,
volatilité, rendements) sont automatiquement calculés et présentés sous
différents formats (tableaux, histogramme, diagrammes …) pour faciliter
l'analyse comparative.

Cette visualisation claire et paramétrable permet à
l'utilisateur d'identifier rapidement les opportunités de marché et d'évaluer
l'efficacité de ses stratégies d'investissement.


\subsubsection{Notions et terminologies de base}

\paragraph{Bourse :} La bourse est un marché organisé où s'échangent des instruments financiers,
tels que des actions, des obligations, des devises, des matières premières, etc. Elle
permet aux entreprises de lever des capitaux en émettant des actions ou des
obligations, et aux investisseurs d'acheter et de vendre ces instruments. Les principales
Bourses mondiales incluent le New York Stock Exchange (NYSE), le NASDAQ, la Bourse
de Paris (Euronext), etc.

\paragraph{Entreprise :} Une entreprise est une entité économique qui produit des biens ou des
services dans le but de les vendre sur un marché. Elle peut être de différentes tailles
(petite, moyenne, grande) et de différents types (privée, publique, coopérative, etc.). Pour notre projet, les
entreprises sont cotées en Bourse, ce qui signifie qu'il est possible d'acheter des actifs de cette 
 dite entreprise. Chaque entreprise devrait avoir un
capital positif.

\paragraph{Action :} Une action est un titre de propriété représentant une partie du capital d'une
entreprise. Lorsqu'un investisseur achète une action, il devient actionnaire de
l'entreprise et a droit à une partie des bénéfices (sous forme de dividendes) qui ne
peuvent être distribués que si l'entreprise réalise des bénéfices. La valeur d'une action
fluctue en fonction de l'offre et de la demande sur le marché boursier.

\paragraph{Obligation :} Une obligation est un titre de créance émis par une entreprise, un État ou
une autre entité pour emprunter de l'argent sur les marchés financiers. L'acheteur d'une
obligation prête de l'argent à l'émetteur en échange d'un intérêt (le coupon) et du
remboursement du principal à une date d'échéance prédéfinie (contrainte d’un contrat).
Les obligations sont généralement considérées comme moins risquées que les actions.

\paragraph{Achat : }L'achat en Bourse consiste à acquérir des titres financiers (actions
ou obligations) en ayant les fonds nécessaire sur le marché
de cette Bourse. L'acheteur devient propriétaire des titres et peut en tirer des bénéfices
(dividendes pour les actions, coupons pour les obligations) ou espérer une plus-value
lors de leur revente.

\paragraph{Vente :} La vente en Bourse consiste à céder des titres financiers que l'on détient
(actions ou obligations) à un autre investisseur sur le marché. L'objectif de la vente est de
réaliser une plus-value (si le titre a pris de la valeur), en évitant de faire  une moins-value (si le titre a
perdu de la valeur).

\paragraph{Actualité : }L'actualité désigne l'ensemble des événements et informations récents ayant
un impact sur la société, l'économie ou la politique. Dans notre contexte de la bourse, elle
englobe les annonces économiques, les décisions politiques et les innovations
technologiques qui influencent les marchés. Sa diffusion continue façonne les
tendances et les comportements des acteurs économiques.









\subsubsection{Contraintes et limitations connues}

\paragraph{Contraintes techniques} 
\paragraph{} Outils de développement:
\begin{enumerate}
\item Langage de programmation : Java
\item IDE : Eclipse
\item Outil de rédaction : Latex
\end{enumerate}

%Un exemple pour l'utilisation des citations bibliographiques
\paragraph{} Nous sommes également contraints de n'utiliser que ce que Le Professeur T.Liu nous transmet dans ses supports pour les CM \cite{data}, en passant outre la documentation sur Internet.


\subsection{Fonctionnalités attendues du projet}
\label{sec:spec2}

\paragraph{Fonctionnalités du programme :} 

\begin{itemize}
    \item Créer N entreprises avec des caractéristiques définies (capital, secteur, taille).
    \item Générer et afficher des événements aléatoires impactant le marché.
    \item Visualiser l'impact de ces événements sur actions, obligations et indices boursiers.
    \item Acheter et vendre des actions sur le marché.
    \item Acheter et vendre des obligations émises par les entreprises.
    \item Visualiser les cours en temps réel et afficher l'évolution historique.
    \item Recevoir des dividendes pour les actions détenues.
    \item Gérer un portefeuille diversifié d’actifs financiers.
    \item Suivre la performance du portefeuille en temps réel (valeur, gains/pertes, dividendes).
    \item Consulter un historique détaillé des transactions et frais de courtage.
\end{itemize}



