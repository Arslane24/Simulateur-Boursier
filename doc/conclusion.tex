\newpage
\section{Conclusion et perspectives}
\label{sec:conclusion}

\subsection{Résumé du travail réalisé}

Ce projet de simulation boursière nous a permis de concevoir et de réaliser un logiciel complet, intégrant à la fois la gestion d'un marché d'actions, un portefeuille utilisateur, et des événements dynamiques influençant les cours.  
Le développement a couvert toutes les étapes classiques d'un projet logiciel : spécifications fonctionnelles, conception de l'architecture, codage modulaire, conception de l'IHM graphique, et mise en œuvre de la logique métier.

Sur le plan technique, nous avons mis en place une architecture orientée objet structurée, facilitant la maintenance et l'évolutivité du code.  
La simulation s'appuie sur un modèle inspiré du mouvement brownien géométrique, adapté pour intégrer des événements économiques variés et un mécanisme de retour à la moyenne.  
Une attention particulière a été portée à l'ergonomie de l'interface graphique afin de garantir une expérience utilisateur fluide et intuitive.

Le travail d'équipe a été essentiel pour mener à bien ce projet, avec une communication régulière, une répartition claire des tâches, et une intégration continue des différentes fonctionnalités.

\subsection{Améliorations possibles du projet}

Plusieurs pistes d'amélioration pourraient être envisagées pour enrichir le simulateur :

\begin{itemize}
    \item \textbf{Ajout de nouvelles classes d'actifs} : intégration d'obligations d'État, de matières premières, ou de crypto-monnaies pour diversifier les investissements.
    \item \textbf{Évolution de l'IA du marché} : implémenter des comportements d'achats/ventes automatiques simulant des investisseurs virtuels pour dynamiser encore davantage le marché.
    \item \textbf{Affinage du modèle économique} : améliorer la prise en compte de la corrélation entre secteurs, intégrer des cycles économiques longs et des crises.
    \item \textbf{Amélioration de l'interface} : proposer des tableaux de bord personnalisables, avec des indicateurs avancés (moyennes mobiles, RSI, volatilité implicite).
    \item \textbf{Système de sauvegarde et de chargement} : permettre à l'utilisateur d'enregistrer la progression de sa simulation et de la reprendre ultérieurement.
    \item \textbf{Mode multijoueur} : proposer des compétitions amicales entre plusieurs utilisateurs pour comparer leurs performances d'investissement.
\end{itemize}

Ces évolutions offriraient un enrichissement considérable de l'expérience utilisateur et prolongeraient l'intérêt de l'application au-delà du cadre académique initial.
