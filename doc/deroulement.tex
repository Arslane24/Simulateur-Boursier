\newpage
\section{Déroulement du projet}
\label{sec:deroulement}

Dans cette section, nous décrivons comment le projet a été réalisé en équipe : la répartition des tâches, la synchronisation du travail en membres de l'équipe, etc.

\subsection{Réalisation du projet par étapes}

Le projet a été réalisé en suivant une démarche progressive structurée autour de plusieurs étapes principales :
\begin{itemize}

\item \textbf{Spécifications fonctionnelles} : définition des besoins du simulateur et rédaction d'un cahier des charges simplifié.
\item \textbf{Conception des classes et de l'architecture} : création des classes principales (\texttt{Entreprise}, \texttt{Action}, \texttt{Obligation}, \texttt{Portefeuille}, etc.) et structuration de l'architecture générale du logiciel.
\item \textbf{Développement de la logique métier} : mise en œuvre des processus de simulation du marché boursier, gestion des événements, et évolution dynamique des cours.
\item \textbf{Création de l'interface graphique} : développement de l'IHM en Java Swing, incluant l'affichage des actifs, des portefeuilles, des graphiques et des événements selon ce que nous avons étudié en cours.
\item \textbf{Tests et validations} : tests réguliers de chaque fonctionnalité ajoutée, corrections de bugs et ajustements ergonomiques au fur et à mesure.
\end{itemize}

\subsection{Répartition des tâches, communication et intégration}

Même si chacun a contribué à l'ensemble du projet, une spécialisation naturelle s’est formée au fil du développement :
\begin{itemize}



\item  \textbf{Arslane} a travaillé sur la logique des processus internes et sur la création et l'intégration des graphiques.
\item  \textbf{Bacem} s'est occupé principalement de la gestion des données (classes d'entités) et de la gestion dynamique des événements économiques.
\item  \textbf{Omar} s'est principalement chargé de la conception de l'interface graphique et de l'ergonomie de l'application.
\end{itemize}


La communication entre les membres a été assurée principalement via Discord, complétée par des échanges en présentiel pendant les cours et des sessions de travail à la bibliothèque.  
Initialement, un dépôt GitHub a été tenté pour le partage du code, mais l'équipe a rapidement basculé sur Discord pour simplifier les échanges et faciliter l'intégration des fichiers.

Des réunions régulières, en présentiel ou en appel, ont permis de suivre l'avancement du projet, de répartir les nouvelles tâches, et de résoudre rapidement les éventuels blocages.
