\newpage
\section{Conception et mise en œuvre}
\label{sec:impl}

\subsection{Architecture globale du logiciel}


\begin{itemize}


    \item \textbf{Package data :} Contient les classes de base représentant les entités du domaine (par exemple, Entreprise,
Action, Obligation, Portfeuille). Ces classes encapsulent les données essentielles et leurs relations,
servant de fondation pour les autres couches du logiciel.


\item \textbf{Package gestion :} Regroupe les classes responsables de la logique métier, telles que la mise à jour des prix
des actifs (Bourse), la gestion des transactions (GestionPortfeuille), et la génération d'événements
influençant le marché (Actualite). Ce package agit comme le moteur du simulateur, en orchestrant les calculs
et les interactions entre les entités.


\item \textbf{Package gui :} Fournit l'interface graphique (IHM) avec des composants Swing pour afficher les données,
interagir avec l'utilisateur, et visualiser les résultats sous forme de tableaux et de graphiques (via JFreeChart). Ce
package assure une expérience utilisateur fluide et intuitive.


\item \textbf{Package test :} Contient la classe Test qui lance l'application en instanciant la fenêtre principale.
\end{itemize}





%Un exemple pour insérer et référencer une figure.
\paragraph{} En résumé dans la figure \ref{fig:architecture} :
\begin{enumerate}
    \item "test" initialise l’application.

    \item "gui" communique avec gestion pour afficher et interagir avec la logique métier.

    \item "gestion" manipule les données de chez "data" (actions, entreprises, événements, portefeuille).
\end{enumerate}

%\begin{figure}
%\centering
%\includegraphics[width=3.5cm, height=2cm]{images/programmer.png}
%\caption{Un programmeur occupé}
%\label{fig:modele}
%\end{figure}

\fig{images/archiglobale.png}{8cm}{7cm}{Schéma de l'architecture globale}{architecture}

\paragraph{}L'architecture suit une approche orientée objet, avec une séparation claire entre les couches de données, de traitement et de présentation. Les fichiers CSV (\texttt{entreprises\_bourse.csv} et \texttt{events.csv}) permettent une extensibilité des données, tandis que l'utilisation de collections Java (comme \texttt{HashMap} et \texttt{ArrayList}) garantit une gestion efficace des entités dynamiques.


\paragraph{}Le flux de données commence par le chargement des entreprises et des événements depuis les fichiers CSV, suivi par des mises à jour périodiques des prix et des événements via la classe \texttt{Bourse}. L'utilisateur interagit avec le système à travers l'IHM, qui affiche les informations en temps réel et permet d'effectuer des transactions.

\subsection{Classes de données}

\paragraph{} Le package data définit les entités fondamentales du simulateur, chacune encapsulant des attributs et des
comportements spécifiques. Voici un aperçu des principales classes de données (Nous ometterons volontairement de noter les getters/setters ):

\fig{images/data.png}{16cm}{11cm}{Package "data"}{data}

\subsection{IHM Graphique}

\paragraph{ Conception et mise en œuvre de l'IHM graphique de votre projet: }


L'interface graphique du simulateur a été conçue pour garantir une expérience utilisateur fluide, en s'appuyant sur une structure claire et modulaire.


L'application repose sur une \textbf{fenêtre principale} qui regroupe l'ensemble des fonctionnalités nécessaires à la gestion d'un portefeuille boursier.  
Elle est organisée autour de plusieurs éléments majeurs :

\begin{itemize}
    \item \textbf{Panneaux d'informations (en haut)} : affichage du solde disponible, du montant investi, du bénéfice/perte et de la santé globale du portefeuille.
    \item \textbf{Barre d'onglets (JTabbedPane)} : navigation entre trois vues principales :
    \begin{itemize}
        \item \textbf{Liste des Actifs} : accès aux actions et obligations disponibles sur le marché.
        \item \textbf{Mon Portefeuille} : suivi des actifs détenus et des performances financières.
        \item \textbf{Graphiques} : visualisation des tendances de marché par entreprise et par secteur.
    \end{itemize}
    \item \textbf{Zone de recherche et de filtres} : située sous les onglets, permettant de filtrer les actifs par secteur ou par événement.
    \item \textbf{Panneau des événements} : affichage des événements économiques impactant le marché.
    \item \textbf{Panneau de contrôle (en bas)} : boutons pour avancer le temps dans la simulation ("Tour suivant"), activer la lecture automatique ("Lecture Auto"), ou créer une nouvelle entreprise.
\end{itemize}

Toutes les informations sont rafraîchies en temps réel pour refléter l'évolution dynamique du marché simulé.

\fig{images/ihm1.png}{12cm}{9cm}{IHM de la fenêtre principale}{ihm}

La figure \ref{fig:ihm} illustre l'organisation générale de l'interface :  
la séparation claire entre les différentes zones fonctionnelles assure une navigation intuitive et une bonne lisibilité des données pour l'utilisateur.

\subsection{Conception des traitements (processus)}
\paragraph{}Les processus du simulateur sont gérés par le package gestion, qui implémente la logique métier pour la simulation du
marché, la gestion des portefeuilles, et la génération d'événements.

Voici les principaux processus :

\paragraph{ Chargement des données :}(Bourse.chargerEntreprises et Actualite.genererlistevenements) 

\begin{enumerate}
    \item \textbf{Description} : Au démarrage, la classe \texttt{Bourse} lit le fichier \texttt{entreprises\_bourse.csv} pour créer des instances de \texttt{Entreprise} (jusqu'à 25 entreprises sélectionnées aléatoirement). La classe \texttt{Actualite} lit \texttt{events.csv} pour générer des événements.
    
    \item \textbf{Mécanisme} : Utilisation de \texttt{BufferedReader} pour lire les fichiers CSV. Parsing des lignes en objets \texttt{Entreprise} et \texttt{Evenement} avec validation des données.
    
    \item \textbf{Impact} : Permet une initialisation flexible du marché avec des données externes.
\end{enumerate}

\paragraph{Mise à Jour du Marché :} (Bourse.Miseajour) 

\begin{enumerate}
    \item \textbf{Description} : Met à jour les prix des actions à chaque tour en fonction de plusieurs facteurs :
    \begin{itemize}
        \item \textbf{Événements} : Impact des événements actifs (via \texttt{Actualite.Miseajour}) sur les secteurs et sous-secteurs.
        \item \textbf{Tendance macroéconomique} : Variation aléatoire globale (basée sur une distribution normale).
        \item \textbf{Volatilité et momentum} : Calculs basés sur les attributs \texttt{volatilite} et \texttt{pourcentage} des entreprises.
        \item \textbf{Retour à la moyenne} : Ajustement des prix pour éviter des écarts extrêmes.
    \end{itemize}
    
    \item \textbf{Mécanisme} :
    \begin{itemize}
        \item Calcul d'un nouveau prix pour chaque entreprise en combinant un terme de dérive (\emph{drift}), un terme stochastique et un facteur de retour à la moyenne.
        \item Mise à jour des attributs \texttt{prixDepart}, \texttt{pourcentage}, \texttt{prixMoyen}, et \texttt{capital}.
        \item Lissage des variations extrêmes (supérieures à 8\%) pour plus de réalisme.
    \end{itemize}
    
    \item \textbf{Impact} : Simule un marché dynamique influencé par des facteurs internes et externes.
\end{enumerate}

\paragraph{}Le code qui permet celà : 
\begin{algorithm}
\caption{Mise à jour du prix d'une entreprise}
\label{alg:update_price}
\begin{algorithmic}[1]
\STATE Calculer $muMomentum \gets$ entreprise.calculerMomentum()
\STATE Calculer $mu \gets 0.25 \times muSecteur + 0.35 \times muEvenement + 0.15 \times muMomentum + 0.25 \times tendanceMacro$
\STATE Calculer $volatilite \gets volatiliteBase + volatiliteEvenement \times 0.8$
\STATE Calculer $termeDrift \gets (mu - \frac{volatilite^2}{2}) \times \Delta t$
\STATE Calculer $secteurEffect \gets$ secteurShock[secteur] (par défaut 0.0)
\STATE Calculer $termeStochastique \gets (0.65 \times volatilite \times \sqrt{\Delta t} \times \mathcal{N}(0,1) + 0.35 \times secteurEffect) \times 0.85$
\STATE Calculer $prixMoyen \gets$ entreprise.getPrixMoyen()
\STATE Calculer $meanReversion \gets 0.03 \times \ln\left(\frac{prixMoyen}{prixInitial}\right) \times \Delta t$
\STATE Calculer $nouveauPrix \gets prixInitial \times \exp(termeDrift + termeStochastique + meanReversion)$
\STATE Calculer $variationPourcentage \gets \frac{(nouveauPrix - prixInitial)}{prixInitial} \times 100$
\IF{$|\text{variationPourcentage}| > 8$}
    \STATE Calculer $facteurLissage \gets \frac{8}{|\text{variationPourcentage}|}$
    \STATE Mettre à jour $nouveauPrix \gets prixInitial \times (1 + \text{sign}(variationPourcentage) \times 0.08 \times facteurLissage)$
\ENDIF
\STATE entreprise.setPrixDepart(nouveauPrix)
\STATE entreprise.setPourcentage($\frac{(nouveauPrix - prixInitial)}{prixInitial} \times 100$)
\STATE entreprise.updatePrixMoyen(nouveauPrix)
\STATE Calculer $capitalInitial \gets$ entreprise.getCapital()
\STATE Calculer $ratioPrix \gets \frac{nouveauPrix}{prixMoyen}$
\STATE entreprise.setCapital($capitalInitial \times ratioPrix$)
\end{algorithmic}
\end{algorithm}



\vspace{0.5cm}

\paragraph{Modèle de calcul des prix :}

La mise à jour des prix des entreprises repose sur une adaptation du modèle du mouvement brownien géométrique (GBM, Geometric Brownian Motion), classiquement utilisé en finance pour modéliser l'évolution stochastique des prix d'actifs.

Le prix $P(t)$ d'une entreprise à l'instant $t$ est actualisé selon la formule suivante :

\[
P(t+\Delta t) = P(t) \times \exp\left(\left(\mu - \frac{1}{2} \sigma^2\right) \Delta t + \sigma \sqrt{\Delta t} \, Z\right)
\]

où :
\begin{itemize}
    \item $\mu$ est le taux de drift (tendance moyenne attendue du prix),
    \item $\sigma$ est la volatilité ajustée,
    \item $\Delta t$ est l'intervalle de temps (ici fixé à 1 tour),
    \item $Z$ est une variable aléatoire suivant une loi normale $\mathcal{N}(0,1)$.
\end{itemize}

Dans notre projet, ce modèle a été enrichi et adapté pour mieux correspondre au contexte d'une simulation ludique :

\begin{itemize}
    \item \textbf{Effets d'événements économiques} : des chocs sectoriels ou macroéconomiques, issus des événements simulés, modifient localement la valeur du drift $\mu$ et de la volatilité $\sigma$.
    \item \textbf{Retour à la moyenne (Mean Reversion)} : une force de rappel est appliquée pour empêcher des écarts de prix trop extrêmes, favorisant une stabilité réaliste autour du prix moyen historique de l'entreprise.
    \item \textbf{Lissage des fortes variations} : si la variation instantanée d'un prix dépasse 8\% en un seul tour, un facteur de correction est appliqué pour adoucir le mouvement et conserver la plausibilité du marché.
\end{itemize}

Ainsi, l'évolution d'un prix au sein du simulateur est déterminée par une combinaison de :

\[
P(t+\Delta t) = P(t) \times \exp\left( \text{Terme de drift} + \text{Terme stochastique} + \text{Retour à la moyenne} \right)
\]

Cette approche hybride permet de conserver un réalisme suffisant tout en introduisant des dynamiques de marché amusantes et variées pour l'utilisateur.




\paragraph{Gestion des Événements :}(Actualite.Miseajour) 
\begin{enumerate}
    \item \textbf{Description} : Gère le cycle de vie des événements, en réduisant leur durée à chaque tour et en générant de nouveaux événements lorsque les anciens expirent.

    \item \textbf{Mécanisme} :
    \begin{itemize}
        \item Décrémentation de la durée des événements actifs.
        \item Suppression des événements dont la durée atteint zéro.
        \item Génération d'un nouvel événement (favorisant les événements positifs avec une probabilité de 60\%) à partir de \texttt{events.csv}.
        \item Ajout de notifications pour informer l'utilisateur des nouveaux événements.
    \end{itemize}
    
    \item \textbf{Impact} : Crée une dynamique de marché en introduisant des perturbations périodiques.
\end{enumerate}

\vspace{0.5cm}

\paragraph{Gestion des Transactions} (GestionPortfeuille)

\begin{enumerate}
    \item \textbf{Description} : Gère les opérations d'achat et de vente d'actions et d'obligations.
    
    \item \textbf{Mécanisme} :
    \begin{itemize}
        \item Achat : Vérification du solde disponible, ajout de l'actif au portefeuille, mise à jour des soldes (\texttt{SoldeActuel}, \texttt{SoldeInvesti}) et enregistrement de la transaction.
        \item Vente : Suppression ou réduction de la quantité d'actifs, mise à jour des soldes, et enregistrement de la transaction.
        \item Utilisation de \texttt{JOptionPane} pour les interactions utilisateur (par exemple, saisie de la quantité).
    \end{itemize}
    
    \item \textbf{Impact} : Permet à l'utilisateur de construire et de gérer un portefeuille en fonction des conditions du marché.
\end{enumerate}

\vspace{0.5cm}

\paragraph{Simulation Automatique} (MainWindow.startAutoSimulation)

\begin{enumerate}
    \item \textbf{Description} : Exécute la simulation en continu avec un intervalle de 1,2 seconde par tour.
    
    \item \textbf{Mécanisme} :
    \begin{itemize}
        \item Utilisation d'un \texttt{Thread} pour appeler \texttt{nextTurn} à intervalles réguliers.
        \item Mise à jour de tous les panneaux (\texttt{ListeActionsPanel}, \texttt{PortfeuillePanel}, \texttt{GraphiquePanel}, \texttt{EvenementsPanel}, \texttt{InfoPanel}) à chaque tour.
        \item Possibilité de mettre en pause ou d'arrêter la simulation via des boutons.
    \end{itemize}
    
    \item \textbf{Impact} : Offre une expérience immersive en simulant un marché en temps réel.
\end{enumerate}

\vspace{0.5cm}

\paragraph{Visualisation des Données} (GraphiquePanel et PortfeuillePanel)

\begin{enumerate}
    \item \textbf{Description} : Affiche les données du marché et du portefeuille sous forme de graphiques.
    
    \item \textbf{Mécanisme} :
    \begin{itemize}
        \item \texttt{GraphiquePanel} : Mise à jour des graphiques linéaires (évolution des prix) et en histogramme (performance par secteur) à chaque tour.
        \item \texttt{PortfeuillePanel} : Mise à jour du graphique en anneau pour la répartition des investissements par secteur.
        \item Utilisation de \texttt{JFreeChart} pour une visualisation professionnelle.
    \end{itemize}
    
    \item \textbf{Impact} : Aide l'utilisateur à analyser les tendances et à prendre des décisions éclairées.
\end{enumerate}

\vspace{0.5cm}

Le simulateur garantit sa robustesse grâce à des vérifications systématiques, notamment sur le solde disponible lors des achats, ainsi qu'à des mécanismes de lissage pour éviter des comportements de marché irréalistes.
Son extensibilité est assurée par l'utilisation de fichiers CSV et la modularité des classes, facilitant l'ajout de nouvelles données.
Enfin, la performance est maintenue grâce à l'utilisation de structures efficaces comme \texttt{HashMap}, optimisant la gestion d'un nombre modéré d’entreprises.









